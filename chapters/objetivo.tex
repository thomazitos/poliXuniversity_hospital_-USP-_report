\documentclass[../delivery_hospital_report.tex]{subfiles}

\begin{document}
\chapter{Objetivo}
No início do projeto, o seu objetivo era desenvolver e construir um robô de entregas que fosse capaz de se navegar de forma autônoma, inicialmente pelo campus da USP, conseguindo interagir com calçadas, pedestres e ciclistas sem a necessidade de intervenção, com o objetivo de realizar entregas. 

Contudo, conforme a pandemia de COVID-19 \cite{covid2020} se intensificou, a necessidade de um robô que exercesse funções parecidas em hospitais se viu necessário, por conta do alto fluxo de pacientes internados com sintomas contagiosos, que além de colocar os funcionários em risco, ainda tornavam o ambiente insalubre. Com isso, se notou a necessidade de construir um robô de delivery hospitalar, que tinha como missão principal transportar medicações e exames sem a necessidade de pessoas o auxiliando. Dessa forma, evitando contato desnecessário. 

\end{document}