\documentclass[../poliXuniversity_hospital_(USP)_report.tex]{subfiles}
%\graphicspath{ {images/}{../images/}{../../images/} }

\begin{document}
A Escola Politécnica e o Hospital Universitário vem se aproximando através de projetos como O Robô Hospitalar, um Robô capaz de se movimentar autonomamente dentro do hospital e realizar as entregas de exames laboratorias. Esse primerio projeto foi promovido pela iniciativa do Professor Leopoldo e Dr. Oscar Fugita, responsável pelo Núcleo de Inovação e Tecnologia do HU(INTEC). Com o sucesso da coolaboração se concretizando através da evolução do projeto, que já se encontrava na Segunda Versão em 2021, foi proposto a resolução de dois novos problemas do HU para esse grupo de alunos responsáveis pelo Robô. No segundo semestre de 2021 o grupo de pesquisa que antes apenas desenvolvia o robô, iniciou a concepção de duas novas máquinas, uma que operaria na Farmácia do HU e outra que seria usada pela equipe de Fisioterapia da UTI Adulto. Como apoio financeiro, o grupo contava com o Amigos da Poli, fundo patrimonial que financiou, desde o inicio, os projetos propostos por Professor Leopoldo. Além do aumento no número de projetos, dos recursos disponíveis veio também um crescimento no número de alunos interessados em inovação na saúde, politécnicos atraidos pelas midias sociais de divulgação do Robo Hospitalar. Em agosto de 2021, tive a oportunidade de, juntamente com meus colegas, fundar a ZIMA - Soluções Médico Hospitalares \cite{ZIMA}, que sintetiza todo esse apoio docente, financeiro e institucional em um grupo de extensão de alunos motivados a impactar a saúde através de tecnologia. Indo além do desenvolvimento de pesquisa, esse documento contempla a criação de um instituição criada por alunos que perpetuará o seu impacto na comunidade USP e na gradução dos Engenheiros Politécnicos oferencendo um espaço de experimentação e aprendizado em inovação da saúde.
\begin{figure}[h]
\centering
    \begin{minipage}{0.3\textwidth}
        \caption{Núcleo de Inovação e Tecnologia (INTEC)}
        \centering % para centralizarmos a figura
        \includegraphics[width=5cm]{images/intec.png}
        \caption*{Fonte: INTEC}
        \label{fig: INTEC}
    \end{minipage}\hfill
    \begin{minipage}{0.3\textwidth}
        \centering
         \caption{Segunda Versão do Robô Hospitalar}
        \centering % para centralizarmos a figura
        \includegraphics[width=5cm]{images/v2_robo_hospitalar.jpeg}
        \caption*{Fonte: autor}
        \label{fig: Robo Hospitalar V2}
    \end{minipage}\hfill
    \begin{minipage}{0.3\textwidth}
        \centering
        \caption{ZIMA - Soluções Médico Hospitalares}
        \centering % para centralizarmos a figura
        \includegraphics[width=5cm]{images/logo_zima.png}
        \caption*{Fonte: autor}
        \label{fig: ZIMA}
    \end{minipage}\hfill
    
\end{figure}


\end{document}