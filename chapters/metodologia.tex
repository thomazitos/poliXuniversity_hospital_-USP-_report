\documentclass[../poliXuniversity_hospital_(USP)_report.tex]{subfiles}

\begin{document}
\chapter{Metodologia}

\subsection{Projetos}

Para concepção, foram realizadas reuniões quinzenais entre alunos, professores e colaboradores do HU, as quais serviam para a definição dos módulos eletrônicos, modelo mecânico e software a ser usado nos projetos. O grupo trabalhou com melhorias incrementais e versionamentos dos equipamentos, consolidando as melhorias tecnicas e corrigindo falhas de projeto. Essas versões foram fabricadas e testadas em laboratório para validação de hipóstese e avaliação da funcionalidade do protótipo. Foi parte do processo de concepção o contato constante com a equipe do Hospital Univesitário para garantir o desenvolvimento de um equipamento útil e funcional.

Toda idealização e prototipagem foi realizada pelos alunos e guiada pelos professores. Foram utilizados softwares de modelagem 3D para o desenvolvimento do CAD de cada projeto, Softwares como Altium Designer na concepção de Placas de Circuito Impresso(PCBs) para os modulos eletronicos e linguagems como C++, python e frameworks como ROS para o desenvolvimento do software responsável pelo funcionamento inteligente dos equipamentos.

Para a aquisição de materiais, contratação de serviços e todos os custo relacionados ao projeto foi usado os recursos doados pelo Fundo Patrimonial Amigos Da Poli, o qual forneceu cerca de XXXX Reais atraves dos seus editais. 

\subsection{Gestão}

Durante todo o processo de desenvolvimento, foi dos professores toda responsabilidade fiscal/fincaceira bem como a comunicação entre Escola Politécnica e HU, enquanto os alunos possuiam liberdade para desenvolver as atividades de projeto em termos de concepção, fabricação e teste, porém com feedback constante dos professores, médicos, farmaceûticos e Fisioterapeutas. Foram estabelecidas reuniões quinzenais entre Escola Politécnica e HU e reuniões semanais para cada área, mecânica, eletrônica e computação feita entre alunos. Além dos encontros oficiais, reuniões extras de alinhamento, visitas técnicas e eventos foram realizandos entre os alunos, professores e equipe do HU durante o processo.

\end{document}