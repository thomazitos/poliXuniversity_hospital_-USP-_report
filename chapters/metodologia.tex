\documentclass[../poliXuniversity_hospital_-USP-report.tex]{subfiles}

\begin{document}
\chapter{Metodologia}

Inicialmente, o robô seria um para entregas por toda a cidade universitária da Universidade de São Paulo. Depois de um tempo, a ideia do projeto foi refeita e ele passou a ser um robô para realizar entregas no ambiente hospitalar. Por conta disso, a primeira versão do robô não estava devidamente adaptada para o lugar que ele estava imerso. Assim, uma segunda versão do robô foi produzida, mas dessa vez, tentando deixar o robô melhor adaptado para o lugar que ele estava imerso. 

Para a parte de Hardware do robô, foram produzidos uma série de protótipos de módulos eletrônicos embarcados, que visam exercer o controle completo do robô, desde motores até mesmo portas e leitura de sensores. Esses módulos começaram a ser idealizados no fim de 2020 e por falta de tempo, as versões oficiais ainda não foram finalizadas.

Para o Software do robô, a princípio, tem como função produzir os algoritmos de controle com auxílio de ROS (Robot Operating System), uma ferramenta muito importante e famosa no escopo da robótica. Dessa área da equipe, cada membro é designado para desenvolver determinado algoritmos e depois integrar com o algoritmos principal.

No começo, para validar os algoritmos de controle, foi feito um ambiente de simulação, que emulava o robô hospitalar em um hospital. Assim, podendo validar os códigos feitos sem entrar em conflito com os eventuais problemas no hardware.

\end{document}